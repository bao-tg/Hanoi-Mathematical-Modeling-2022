\section{Đánh giá về mô hình}
\subsection{Ưu điểm}
\begin{itemize}
    \item Các mô hình không quá phức tạp, dễ tính toán.
    \item Các công thức xây dựng trong bài có thể không mang lại kết quả đúng nhất nhưng đơn giản và dễ làm. Trên thực tế, đối với việc mô hình hóa bài toán, thay vì mất nhiều thời gian để xử lý bài toán một cách chính xác gần như tuyệt đối thì việc sử dụng các thuật toán đơn giản hơn dù không ra được kết quả tốt nhất vẫn được sử dụng phổ biến.
    \item Có tính thực tiễn.
    \item Đối với sức khỏe
    \begin{itemize}
        \item Đảm bảo được sự đa dạng của thực đơn với đầy đủ các món mà không ảnh hưởng chất lượng dinh dưỡng.
        \item Đảm bảo cân bằng giữa hai yếu tố giá trị dinh dưỡng và nhu cầu cần thiết.
    \end{itemize}
    \item Đối với giáo dục và đào tạo
    \begin{itemize}
        \item Mục tiêu, định hướng rõ ràng, đơn giản.
        \item Tính toán thời gian hợp lý đảm bảo việc học không ảnh hưởng đến các hoạt động khác.
    \end{itemize}
    \item Đối với thu nhập
    \begin{itemize}
        \item Cách làm dễ dàng, phù hợp với nhiều đối tượng.
        \item Vừa giúp kiếm được thu nhập, vừa giúp nâng cao kĩ năng của bản thân.
        \item Có thể linh hoạt chọn cho những công việc hợp với bản thân qua danh sách gợi ý những việc làm có thể làm tại nhà trong mùa dịch.
    \end{itemize}
\end{itemize}
\subsection{Nhược điểm}
\begin{itemize}
    \item Các công thức không chính xác hoàn toàn mà chỉ mang tính tương đối.
    \item Các plan chỉ phù hợp cho một số đối tượng đã giả sử nhất định.
    \item Plan về thực đơn ăn uống chưa xét đến các yếu tố như khẩu vị của người ăn, độ ưa thích của các món. Số lượng kcal cung cấp cho cơ thể không phù hợp với nhiều đối tượng. Chưa tính đến giá cả các món ăn.
    \item Plan về học tập áp dụng cho những bạn trẻ chuẩn bị bước vào đại học nên phụ thuộc nhiều vào trình độ, mục đích của đối tượng được lựa chọn. Chưa nói đến những khó khăn trong quá trình học tập.
    \item Plan về thu nhập chưa được liên kết với kế hoạch học tập. Trên thực tế, hai điều này có thể ảnh hưởng lẫn nhau về mặt thời gian.
\end{itemize}
\subsection{Cải thiện mô hình}
\begin{itemize}
    \item Trong phần thực đơn ăn uống, ta có thể xét thêm những khía cạnh về dinh dưỡng khác của món ăn như chất đạm, protein, chất xơ. Dựa vào khẩu vị và thông tin trên mạng có thể đánh giá và mô hình hóa lượng dinh dưỡng và năng lượng bị hao hụt trong quá trình chế biến, khả năng hấp thụ với các món khác nhau tùy theo khẩu vị. Ngoài ra, có thể tính đến giá thành của các món ăn nhằm mang lại hiệu quả tốt nhất, giảm bớt chi phí.
    \item Có thể đưa ra nhiều phương pháp học tập, lập kế hoạch cụ thể hơn (theo từng tuần hoặc từng ngày).
    \item Lập thời gian biểu theo các khung giờ trong ngày cho việc học và cả việc làm thêm kiếm thu nhập.
\end{itemize}
